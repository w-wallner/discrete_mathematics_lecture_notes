
\section{Generating Functions}

Generating functions provide a tool for coping with combinatorial enumeration problems. The ordinary generating function defines the sum of a sequence $a_n$:
\[
	(a_n)_{n≥0} = \sum_{n≥0} a_n z^n \text{, formal power series}
\]
\paragraph{Operations on generating functions:}
\begin{itemize}
	\item Two power series can be added:
	\[
		\sum_{n≥0} (a_n + b_n) z^n \text{,}
	\]

	\item multiplied:
	\[
		\sum_{n≥0}\sum_{k=0}^n a_k b_{n-k} z^n \qquad \text{(Cauchy product),}
	\]

	\item and if $b_0 ≠ 0$, we can also divide a power series by another power series:
	\[
	    \frac{
	        \sum a_n z^n
	    }{
	        \sum b_n z^n
	    } =
	        \sum c_n z^n
	\]
\end{itemize}

If we have
\[
    (*) = \sum_{n≥0} a_n z^n = \lim_{n \to ∞} \sum_{k=0}^{n} a_k z^k
\]

and $(*)$ is convergent, then the domain of convergence is a disk with center (0,0) and the radius of convergence is $R$:
\[
    R = \frac1{\overline{\lim} \sqrt[n]{|a_n|}} \in [0, \infty]
\]

\Theorem.
Let
\[
    f(z) = \sum a_n (z-z_0)^n,
    a_i\in \mathbb{C},
    R = \frac{1}{\overline{\lim} \sqrt[n]{|a_n|}}
\]
Then
\begin{enumerate}[a)]
	\item $|z - z_0| < R \implies f(z)$ absolutely convergent, i.e.
	      $\sum |a_n| (z-z_0)^n$ convergent

	\item $|z - z_0| > R \implies f(z)$ divergent
\end{enumerate}

\textbf{Examples.}
\begin{align*}
    &\sum_{n≥0} z^n =
        \frac1{1-z}\quad\text{for $|z|<1$, } R = 1 \\
    &\sum_{n≥0} \frac{z^n}{n!} =
        e^z, ~R = ∞ \\
    &\sum_{n≥0} n! z^n, ~R = 0
\end{align*}
Inside the disk of convergence, we even have \emph{uniform convergence} (allows interchange of limits).
\begin{align*}
    \frac1{(1-z)^2}
    &= \left(\frac1{1-z}\right)'
    = \left(\sum_{n>=0} z^n\right)'
    = \sum_{n≥1} n z^{n-1} \\
    \log\frac{1}{1-z} &= \sum_{n≥1} \frac{z^n}{n}
\end{align*}

\Theorem. (Identity theorem for power series).
Let
\[
    f(z) = \sum_{n≥0} a_n (z-z_0)^n,
    \text{f(z) convergent for $|z-z_0| < \epsilon$}
\].
The coefficients $a_n$ are unique and satisfy
\[
    a_n = \frac{f^{(n)}(z_0)}{n!}.
\]
$f(z)$ is the \emph{Taylor series}.

\textbf{Corollary.}
\[
    \sum a_n (z-z_0)^n =
    \sum b_n (z-z_0)^n
    \text{ for } |z-z_0| < \epsilon
    \implies
    a_n = b_n.
\]

Since $f(z)$ generates the sequence $(a_n)$ by continued differentiation and evaluation, we call $f(z)$ the \dt{generating function}. In particular,
\begin{align*}
\sum a_n z^n & \text{ ordinary generating function} \\
\sum a_n \frac{z^n}{n!} & \text{ exponential generating function.}
\end{align*}


\subsection{Operations On Generating Functions}

Let sequence $(a_n)_{n≥0}$ correspond to the generating function $\displaystyle{\sum_{n≥0} a_n z^n = A(z)}$ and $(b_n)_{n≥0}$ correspond to $B(z)$.

\begin{itemize}
\item Addition
\begin{align*}
    (\alpha a_n + \beta b_n)_{n≥0}
        &\leftrightarrow \alpha A(z) + \beta B(z)
        &&\forall \alpha, \beta \in \mathbb{C}
        \quad\text{(Linearity)}
\end{align*}
\item Multiplication
\begin{align*}
    \left(\sum_{k=0}^{n} a_k b_{n-k} \right)
        &\leftrightarrow A(z) B(z),
        \text{ in particular: }
        \left(\sum_{k=0}^n a_k\right)_{n≥0} \leftrightarrow \frac{1}{1-z} A(z)
\end{align*}

\Remark.
\begin{align*}
    \hat{A}(z) = \sum_{n \geq n} a_n \frac{z^n}{n!}, \\
    \hat{B}(z) = \sum_{n \geq n} b_n \frac{z^n}{n!}, \\
    \left(\sum_{k=0}^n \binom{n}{ k} a_k b_{n-k}\right)_{n \geq 0}
        \leftrightarrow \hat A(z)\hat B(z)
\end{align*}

\item $
    (a_n \gamma^n)_{n≥0}
        ↔ A(\gamma z) $
\item $ (a_{n-1})_{n≥1}↔ z A(z)$
\begin{align*}
        (a_{n+1})_{n≥0} ↔ \frac{A(z)-a_0}z,
        EGF: (a_{n+1})_{n≥0} ↔ \hat A'(z) \\
\end{align*}
\item $(n a_n)_{n≥0}↔ z A'(z)$

\end{itemize}

\textbf{Example.}
\[
    \sum_{n≥0} (-1)^n z^n
    = \frac{1}{1+z} \quad\text{for $|z| < 1$}.
\]

\textbf{Example.}
\[
    \sum_{n≥0} n z^n
    = \frac{z}{(1-z)^2}
    = z \left(\frac1{1-z}\right)'.
\]

\textbf{Example.}
\[
    \sum_{n≥0} \binom{\alpha}{n} z^n
    = (1+z)^\alpha
    \quad\forall\alpha\in\mathbb{C}
\]

\textbf{Example.}
\begin{align*}
    a_n &= \sum_{k≥0}^n k\\
    \sum_{n≥0} a_n z^n
    &= \sum_{n≥0}
        \left(\sum_{k=0}^n k\right) z^n
    = \sum_{n≥0} \left(\sum_{k=0}^n  k\cdot 1\right) z^n
        = \left( \sum_{n≥0} n z^n \right) \left( \sum_{n≥0} 1 \cdot z^n \right)\\
    &= \frac{z}{(1-z)^3}
    = \frac12 z \left(\frac{1}{1-z}\right)''
    = \frac z2 \sum_{n≥2} n(n-1) z^{n-2}\\
    &= \frac z2 \sum_{n≥0} n(n-1) z^{n-2}
    = \sum_{n≥0} \frac{(n+1)n}{2} z^n
    = \sum_{n≥0} {\binom{n+1}{2}} z^n
\end{align*}

\Lemma.
\begin{gather*}
    \sum_{n≥0} {\binom{n+k-1}{k-1}} z^n
    = \frac1{(1-z)^k} \\
    (1 + z)^\alpha = \sum {\binom{a}{n}} z^n \\
\end{gather*}

Sketch of proof:
\[
    {\binom{n+k-1} {k-1}} = \ldots = (-1)^k{\binom{-k}{n}}
\]


\subsection{Recurrence relations}

\textbf{Example.}
Consider the Towers of Hanoi problem with $n$ disks. How many steps do we need to move the disks?

\begin{figure}[htbp]
  \centering  \includegraphics[width=0.8\textwidth]{02_higher_combinatorics/pics/TowerOfHanoi}
  \caption{Solution for the Tower of Hanoi with 2 disks}
\end{figure}
\begin{align*}
  a_0 = 0 \\
  a_1 = 1 \\
\end{align*}
For the $n+1$-th step, put $n$ disks to a temporary location, move the $n+1$-th disk, move the $n$ disks again.
\begin{align*}
  a_{n+1} &= 2 a_n + 1, a_n = 2^n - 1\\
  A(z) &= \sum_{n \geq 0} a_n z^n
\end{align*}

We multiply both sides with $z^{n+1}$ and sum up over $n$.

\begin{align*}
\sum_{n≥0} a_{n+1} z^{n+1} &= 2 \sum_{n \geq 0} a_n z^{n+1} + \sum_{n\geq 0} z^{n+1} \\
A(z) - \underbrace{a_0}_{0} &= 2z A(z) + \frac{z}{1-z}\\
A(z) &= 2 z A(z) + \frac{z}{1-z}\\
A(z) &= \frac{z}{(1-z)(1-2z)} \\
    &= \frac{\alpha}{1-z} + \frac{\beta} {1-2z}\\
    &= \frac{-1}{1-z} + \frac{1}{1-2z} \\
    &= -\sum_{n\geq 0} z^n + \sum_{n\geq 0} 2^n z^n\\
    &= \sum_{n\geq 0} (2^n-1) z^n\\
\end{align*}

\textbf{Example.}
\begin{align*}
    F_0 = 0, F_1 = 1, F_{n+2} = F_{n+1} + F_n \\
    F(z) &= \sum_{n≥0} F_n z^n : \\
    F(z) - \underbrace{F_0}_{0} - F_1 z &= z \left( F(z) - \underbrace{F_0}_{0} \right) + z^2 F(z)\\
    \implies F(z) &= \frac{z}{1-z-z^2} \\
    F(z) &= \frac{-z}{(z-z_1) (z-z_2)},
        \quad z_{1,2} = \frac{-1 \pm \sqrt{5}}{2} \\
    F(z) &= \frac1{\sqrt{5}} \cdot \frac{1}{1- \frac{1 + \sqrt{5}}{2} \cdot z} - \frac{1}{\sqrt{5}} \cdot \frac{1}{1- \frac{1 - \sqrt{5}}{2} \cdot z}  \\
    \implies F_n &= \frac{1}{\sqrt{5}}
		\left( \left( \frac{1+ \sqrt 5}{2} \right)^n
		- \left( \frac{1-\sqrt{5}}{2} \right)^n
		\right)
\end{align*}

In general:
\[
    a_{n+k} + q_1 a_{n+k-1} + \cdots + q_k a_n = 0
    \quad  \text{ (*) for }n \geq 0
\]
(*) is unique if the first $k$ elements are given

$a_0, ..., a_{k-1}$ given, $q_i$ are given constants

\begin{align*}
A(z) = \sum_{n\geq 0} a_n z^n
\end{align*}
\begin{align*}
\sum_{n\geq 0} a_{n+k} z^{n+k} + q_1 \sum_{n\geq 0}  a_{n+k-1}z^{n+k} + \ldots + q_k \sum_{n\geq 0} a_k z^{n+k} = 0
\end{align*}
\begin{align*}
A(z) - a_0 - a_1 z - \ldots - a_{k-1} z^{k-1} + q_1 z \left( A(z) - \sum_{i=0}^{k-2} a_i z^i \right) + \ldots + q_k z^k A(z) = 0\\
A(z) = \frac{ p(z) }{ 1 + q_1 z + q_2 z^2 + \ldots + q_k z^k } = \frac{p(z)}{q(z)} = \frac{p(z)}{\prod_{i=1}^{r} (z - z_i)^{\lambda_i}}
\end{align*}

Observation: $deg(p) < deg(q)$.

Use ansatz:
\begin{align*}
\frac{p(z)}{q(z)} &= \sum_{i=1}^{r} \sum_{j=1}^{\lambda_i} \frac{A_{ij}}{(z-z_i)^j} \\
&= \frac{A_{11}}{z-z_1} + \frac{A_{12}}{(z-z_1)^2} + \cdots + \frac{A_{1\lambda_i}}{(z-z_1)^{\lambda_i}} + \cdots \\
&= \sum_i\sum_j \frac{B_{ij}}{\left(1-\frac{z}{z_i}\right)^j}= \sum_i\sum_j B_{ij} \cdot \binom {n+j-1}{j-1} \cdot z_i^{-n}\\
&= \frac{\alpha}{1-z} + \frac{\beta}{1-2z} + \frac{\gamma}{(1-2z)^2} + \frac{\delta}{(1-2z)^3}\\
&\implies A_z = \sum_{n\geq 0}\left(p_1(n) \left( \frac{1}{z_1} \right)^n + p_2(n) \left( \frac{1}{z_2} \right)^n + \cdots + p_i(n) \left( \frac{1}{z_i} \right)^n \right)z^n \\
&deg(p_i) \leq \lambda_i + 1
\end{align*}
\TODO{Check the previous formulas for completeness}
\TODO{Complement with Patrik's notes}

\textbf{Example.}
\[
    a_{n+2} - 4 a_{n+1} - 4 a_n = 0 \quad n ≥ 0;  a_0, a_1 \text{ given}
\]
\[
    A(z) = \sum a_n z^n
\]
\begin{align*}
    A(z) - a_0 - a_1 z - 4z (A(z)-a_0) + 4z^2 A(z) &= 0 \\
    (1-4z + 4z^2) A(z) &= a_0 + a_1 z - 4a_0 z \\
    A(z) &= \frac{a_0 + (a1 - 4a_0)z}{1- 4z + 4z^2}\\
    &= \frac{a_0 + (a1 - 4a_0)z}{(1-2z)^2}\\
    &= \frac{C}{1-2z} + \frac{D}{(1-2z)^2} = (*)\\
    \implies a_0 + (a_1 - 4a_0)z &= C (1-2z) + D \\
    [z^0]: a_0 &= C+D \\
    [z^1]: a_1 - 4a_0 &= -2C \\
    \implies C, D
\end{align*}
\begin{align*}
    (*) = C \cdot \sum_n 2^n z^n + D \cdot \sum_n (n+1) \cdot 2^n z^n \\
    = \sum_n \left(2^n \cdot C + (n+1)\cdot 2^n D\right) z^n \\
    = \sum_n (a_n) z^n
\end{align*}

\TODO{Add inhomogeneous recurrence relation to above example, i.e., $0$ on the rhs of the initial equation is replaced by $f(n)$}

\subsection{Unlabeled Combinatorial Structures}

\textbf{Example.}
Binary tree (complete), (without cycles, plane, rooted, either no further children (external nodes, leaves) or 2 children (internal nodes))

\begin{figure}[htbp]
  \centering
    \includegraphics[width=0.7\textwidth]
      {02_higher_combinatorics/pics/BinaryTree}
  \caption{A binary tree}
\end{figure}

A binary tree is a plane structure.

\begin{center}
  \includegraphics[width=0.7\textwidth]
    {02_higher_combinatorics/pics/BinaryTreeUnequal}
\end{center}

$a_n$ = number of binary trees with $n$ internal nodes

If there are n internal nodes, then there are n+1 leaves. The number of vertices in a binary tree is always odd ($n + (n+1)$). We try to describe the binary tree recursively.

\begin{center}
  Binary tree with size $n+1$ ($k$ ranges from $0$ to $n$):
  \includegraphics[width=0.5\textwidth]
    {02_higher_combinatorics/pics/BinaryTreeRecursion}
\end{center}
\begin{itemize}
  \item The left child has size $k$
  \item The right child has size $n-k$
\end{itemize}
\begin{align*}
    a_0 &= 1 \\
    a_{n+1} &= \sum_{k=0}^{n} a_k a_{n-k}
        \quad | \cdot z^{n+1} | \sum \\
    A(z) - 1 &= z A(z)^2
        \quad \text{(Cauchy-Product)} \\
    A(z) &= \frac{1 \pm \sqrt{1-4z}}{2z}\\
        &= \frac{1 \pm (1-4z)^\frac1{2}}{2z} \qquad
    \text{“$+$” is not a viable option, so we use “$-$”} \\
     \\
    (1 + z)^\alpha &= \sum_n {\binom{\alpha}{ n}} z^n \\
    (1-4z)^\frac1{2} &= \sum_{n \geq 0} \binom{\frac12}{n} \cdot (-4)^n \cdot z^n\\
	&= - \sum_{n\geq 0} \frac{\frac12 (-\frac12) (-\frac32) \ldots (\frac12 - n+1))}{n!} \cdot (-4)^n z^n \\
    &\sum_{n\geq 0} \underbrace{\frac{1}{n+1} {\binom{2n}{n}}}_{\text{ Catalan numbers}} z^n
\end{align*}

A binary tree of even size has a certain number of internal nodes: $\displaystyle{a_n = \frac{1}{n+1} {2n \choose n}}$\\
$A(z) = 1 + zA(z)^2$

We look at the set of binary trees $B$. The tree is either a leaf with size 0, or a node with two binary trees as children.

\textbf{Examples.}

\textbf{Example 1.}
We have a bin with red, blue and yellow balls. We have 2 or 3 red balls, at least one blue ball, and not more than one yellow ball. We now ask for the number of combinations with $n$ balls, conforming to the given constraints.

Assume we introduce a variable for each color: $r,b,y$. We count by the exponent. So our function is:
\[
    1\cdot r^2 + r^3,  1+y, b + b^2 + \ldots = \frac{b}{1-b}
\]

We now have 2 configurations:
\[
    A(z) = \sum a_n z^n \quad
    B(z) = \sum b_n z^n
    \implies \sum_{k=0}^{n} a_k b_{n-k}
\]

This leads to the formula:
\[
    (r^2  z^2+ r^3 z^3) (1+zy) \left(\frac{bz}{1-bz}\right)
    = \sum a_{lmkn} r^l b^m y^k z^n
\]

The coefficient $a_{lmkn}$ is the number of configurations with l red, m blue, k yellow and a total number of n balls

We are not interested in the number of red, blue or yellow balls, but the total number of used balls, so we set $r=b=y=1$ and simplify the formula to:
\begin{align*}
    (z^2 + z^3)&(1+z) \frac{z}{1-z} \\
    \implies \text{\# comb}
    &= \left[z^n\right] \frac{z^3(1+z)^2}{1-z} \\
    &= \left[z^{n-3}\right] \frac{1}{1-z}+ 2\left[z^{n-4}\right] \frac{1}{1-z} + \left[z^{n-5}\right] \frac{1}{1-z} \\
    a_n &= 4 \quad \text{for } n ≥ 5 \quad (a_3=1, a_4=3)
\end{align*}

\textbf{Example.}
Number of combinations w/o repetition of a set $M = \{1,2, \ldots ,N\}$. We now want to know the number of combinations of size $k = {N \choose k}$.

\[
    a_1, \ldots , a_n \mathrel{\hat{=}} \text{different balls} \mathrel{\hat{=}} \text{elements of } M
\]

\[
    % a_1^0, a_1^1
    (1+a_1)(1+a_2)(1+a_3) \ldots (1+a_N)
\]

We now need to consider the size of the set
\[
    (1+x)(1+x)(1+x) \ldots (1+x)^N = \sum {N \choose k} x^k
\]

If we allow repetition:
\[
    \prod _{i=1}^N (1+a_i + a_i^2 + \ldots) = \prod _{i=1}^N \frac{1}{1-a_i} \quad a_i = x
\]

\[
    f(x) = \prod _{i=1}^N \frac{1}{1-x} = \frac{1}{(1-x)^N}
    = (1-x)^{-N}
    = \sum _{k \geq 0} {-N \choose k} (-1)^k x^k
    = \sum _{k \geq 0} {N + k-1 \choose k} x^k
\]

\subsection{Combinatorial construction}
$\mathcal{A}$ is a combinatorial class, which is a set of objects.
Furthermore there is a size-function $w: \mathcal{A} \rightarrow \mathbb{N}$.

\[
a_n = \left(\text{number of objects } x\in \mathcal{A}\right) \text{ such that }w(x) = n < \infty,  \forall n \in \mathbb{N}
\]

generating function of $(\mathcal{A}, w)$
\[
    A(z) = \sum_{n\geq 0 } a_n z^n
\]

We now create combinatorial classes from combinatorial classes.
$(\mathcal{A}, w_A) , (\mathcal{B}, w_B)$

\begin{enumerate}[1)]
\item combinatorial sum $\mathcal{A} + \mathcal{B}$:
    we assume that $\mathcal{A} \cap \mathcal{B} = \varnothing$
    \begin{align*}
        \mathcal{A} + \mathcal{B} &= (\mathcal{A} \cup \mathcal{B} , w), \\
        w(x) &= \begin{cases}
            {w_A(x) \quad x \in \mathcal{A}}\\
            {w_B(x) \quad x \in \mathcal{B}}
			\end{cases}\\
        c_n &= a_n + b_ n\rightarrow C(z) = A(z) + B(z)
    \end{align*}
\item
    \begin{align*}
        \mathcal{A} \times \mathcal{B}
        &:= (\mathcal{A} \times \mathcal{B}, w)
        \qquad \text{with } w((x,y)) = w_a(x) + w_b(y), x\in \mathcal{A}, y\in \mathcal{B}\\
        c_n &= \sum_{k=0}^n a_kb_{n-k}\\
        C(z) &= A(z)B(z)
    \end{align*}

\item
    \begin{align*}
        \seq(\mathcal{A}) &= \{ (x_1, x_2, \ldots x_k) \mid k \in \mathbb{N}_0, x_i \in \mathcal{A}\}
        \qquad k = 0 \mathrel{\hat{=}} \epsilon \ldots \text{empty sequence}\\
        &w((x_1, x_2, \ldots x_k) ) = \sum_{i=1}^{k} w_A(x_i) \\
        \mathcal{C} &= \seq(\mathcal{A} = \{\epsilon\} \cup \mathcal{A} \cup (\mathcal{A}\times\mathcal{A}) \cup \ldots )\\
        C(z) &= 1 + A(z) + A(z)^2 + \ldots = \frac{1}{1-A(z)} \\
    \end{align*}

    Assumption $a_0 = 0$
\end{enumerate}

\textbf{Example.}
Integer partitions:

Decomposition of an integer into a sum of smaller integers ($5 = 3+1+1= 1+3+1 = 2+2+1 = 1+1+1+1+1$). The order does NOT matter ($5= 4+1 = 3 + 2 = 3+1+1 = 2+2+1 = 2+1+1+1 = 1+1+1+1+1$).

If the order matters, then we speak of \dt{compositions} of an integer

\begin{align*}
    \mathcal{C} &\ldots \text{compositions of an integer }\in \mathbb{N}^{+} \\
    \mathcal{C} &= \seq(\mathcal{A}) \\
     \mathcal{A} &= \mathbb{N}^{+}, w_A(x) = x\\
	\underline{x} &= (x_1, x_2, \ldots x_k) \in \mathcal{C}\\
    \underline{x} &\mathrel{\hat{=}} x_1 + x_2 + \cdots + x_k\\
    w(\underline{x}) &= x_1 + x_2 + \ldots x_k\\
    \mathcal{A}&=\seq( \{ 0 \})\setminus \{\epsilon\}) \\
    A(z) &= \frac{1}{1-z} - 1 = \frac{z}{1-z} \\
    C(z) &= \frac{1}{1-A(z)} = \frac{1}{1-\frac{z}{1-z} }
         = \frac{1-z}{1-2z}
         = 1 + \sum_{n\geq 1} 2^{n-1} z^n \\
    P(z) &= \prod_{i\geq 1} \frac{1}{1-x^i}\\
	\mathcal{P} &= \seq(\{1\}) \times \seq(\{2\}) \times \cdots \\
\end{align*}

\textbf{Example.}
Strings composed of $\cdot, -$.
“$\cdot$” has length 1 and “$-$” has length 2.
We want to compute the number of strings of length $n$
\begin{align*}
  \mathcal{A} &= \{\cdot\} && \leftrightarrow z\\
  \mathcal{V} &= \{ - \} \mathrel{\hat{=}} \mathcal{A} \times \mathcal{A} && \leftrightarrow z^2
\end{align*}

\textbf{Example.}
Binary trees, as explained before
\begin{align*}
  \mathcal{B} &= \{ \Box\} + \{ \circ\} \times \mathcal{B} \times \mathcal{B}\\
  B(z) &= 1 + z \cdot B(z) \cdot B(z)
\end{align*}

\subsection{Labelled constructions}
\textbf{Example.}
Permutations, $a_k = n!$
\[
  A(z) = \sum_{n\geq 0} n! z^n
\]

Cyclic permutations: $(p_1 p_2 \ldots p_n) \mathrel{\hat{=}} (p_3 p_4 \ldots p_n p_1 p_2)$

The number of cyclic permutations is therefore $(n-1)!$

With the generating function:
\begin{align*}
    B(z) = \sum (n-1)! z^n
\end{align*}

\begin{definition}
We define a \dt{labelled structure} $\mathcal{A}$ in the following way: Each object of size $n$ is composed of $n$ atomic objects. The atoms are numbered with numbers from $1$ to $n$.
\end{definition}
$A \leftrightarrow \text{generating function } \hat{A}(z) = \displaystyle{\sum_{n\geq 0} a_n \frac{z^n}{n!}}$
\begin{align*}
    \mathcal{C} &= \mathcal{A}+\mathcal{B}: \\
         w(x) &= \begin{cases}
                w_A(x) \quad x\in \mathcal{A}\\
                w_B(x) \quad x\in \mathcal{B}
				\end{cases}\\
    \hat{C}(z) &= \sum_{n\geq 0} (a_n+b_n) \frac{z^n}{n!}
        = \hat{A}(z) + \hat{B}(z)
\end{align*}

\begin{definition}
$\mathcal{A} * \mathcal{B}$ \dt{partitional product}\\
\begin{align*}
  \mathcal{A} * \mathcal{B} = \{ (x,y) | x \in \mathcal{A}, y \in \mathcal{B},
    &\text{ atoms are labelled in order preserving way }\\
    &\text{ s.t. the labels are } 1,2, \dotsc , w_A(x) + w_B(y) \}
\end{align*}
\end{definition}

\begin{align*}
    c_n = \sum_{k=0}^{n} {n \choose k} a_k b_{n-k}
\end{align*}

\TODO{figure of $k$ elements $\in \mathcal{A}$}

\begin{align*}
    \frac{c_n}{n!} &= \sum_{k=0}^{n} \frac{1}{k!(n-k)!} a_k b_{n-k}\\
    &= \sum_{k=0}^{n} \frac{a_k}{k!} \frac{b_{n-k}}{(n-k)!} \\
    &[z^n] \hat{A}(z)\hat{B}(z)
        \quad [z^k] \hat{A}(z)
        \quad [z^{n-k}] \hat{B}(z) \\
    &\implies \hat{C}(z) = \hat{A}(z) \hat{B}(z) \\
\end{align*}

\begin{align*}
    \seq(\mathcal{A}) &= \{\epsilon\} \times \mathcal{A} \times \mathcal{A}*\mathcal{A} \times \ldots \\
    \hat{C}(z) &= \frac{1}{1-\hat{A}(z)} \\
    \set(\mathcal{A}) &\mathrel{"\cong"} \{\varnothing\}
        \times \mathcal{A}
        \times \frac{1}{2}\mathcal{A} * \mathcal{A} + \frac{1}{3!}\mathcal{A}*\mathcal{A}*\mathcal{A} \\
        \implies \hat{C}(z) &= e^{\hat{A}(z)}
\end{align*}

\textbf{Side-remark.} In the unlabeled case:
\begin{align*}
  \set(\mathcal{A}): C(z) = \exp\left( A(z) - \frac{A(z^2)}{2} + \frac{A(z^3)}{3} + \ldots \right)
\end{align*}

cycle of $\mathcal{A}$
\begin{align*}
    \cyc(\mathcal{A}) &\cong \mathcal{A} + \frac{1}{2}\mathcal{A} * A + \frac{1}{3}\mathcal{A}*\mathcal{A} + \mathcal{A} + \mathcal{A} + \ldots \frac{1}{k}\mathcal{A}^k \\
    \hat{C}(z) &= \log\frac{1}{1-\hat{A}(z)}
\end{align*}

\textbf{Example.}
Permutations.
\begin{align*}
    \mathcal{P} &= \text{set}(\text{cyc}(\{ \circ \} )) \\
    \hat{p}(z) &= e^{\log \frac{1}{1-z}} = \frac{1}{1-z} \\
        &= \sum_{n\geq 0} n! * \frac{z^n}{n!}
\end{align*}

\textbf{Repetition of last time.}

Labelled structure set $\mathcal{A} = (A,w)$, the atomic objects can be enumerated.
\begin{align*}
  \mathcal{A}+\mathcal{B} &\leftrightarrow \hat{A}(z) + \hat{B}(z) \\
  \mathcal{A}*\mathcal{B} &\leftrightarrow \hat{A}(z) * \hat{B}(z) \\
  \seq(\mathcal{A}) &\leftrightarrow \frac{1}{1 - \hat{A}(z)}\\
  \set(\mathcal{A}) &\leftrightarrow \exp(\hat{A}(z)) \\
  \cyc(\mathcal{A}) &\leftrightarrow \log \frac{1}{1 - \hat{A}(z)}\\
\end{align*}
% end of repetition
\textbf{Example.}
\begin{align*}
  \mathcal{P} &= set(cyc(\mathcal{A})), \mathcal{A} = \{ \text{\ding{172}} \}\\
  \hat{P}(z)  &= \exp{(\log{\frac{1}{1-z}})} = ∑_{n≥0} n! \frac{z^n}{n!}
\end{align*}

\textbf{Example.}
We have set partitions $M = M_1 \cup M_2 \cup \ldots \cup M_k$ with $M_i \neq \varnothing$ and $M_i \cap M_j = \varnothing$
\begin{align*}
  \mathcal{P} = \set( \set( \mathcal{A}) \backslash \{\varnothing\} ) \\
  \hat{P}(z) = \exp(\exp((z) - 1)) \\
  [z^n] e^{e^z - 1} = \sum_{k\geq 0 } S_{n,k} \\
\end{align*}

\subsection{Exponential Generating Functions and Ordered Structures}

Ordered n-tuples $(q_1,q_2, \ldots, q_n), q_i \in \{1,2, \ldots, N\}$, no repetition

$a_n$ is number of such n-tuples

$Q = \underbrace{\{\epsilon,1\}}_{1+z} *
     \underbrace{\{\epsilon,2\}}_{1+z} * \ldots *
     \underbrace{\{\epsilon,N\}}_{1+z} $

\begin{align*}
  \sum_{n\geq 0} a_n \frac{z^n}{n!} &=
    (1+z)^N \\
    &= \sum_{n=0} ^{N} {N \choose n} z^n \\
    &= \frac{N!}{(N-n)!}
\end{align*}

With repetition:
\[
  (1+z+ \frac{z^2}{2!} + \frac{z^3}{3!} + \ldots)^N
    = e^{zN} = \sum_{n\geq 0} N^n \frac{z^n}{n!}
\]
