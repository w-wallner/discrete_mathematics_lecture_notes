
\section{Congruence Relations and Residue Classes}

\begin{definition}
$m \in \mathbb{N}^{+}$ modulus,
residue class: $a+m \mathbb{Z} = \bar{a}$
\end{definition}

\Remark.
\[
  a\in \bar{a},\; \bar{a} \equiv \bar{b} \iff m | a-b
\]
\begin{align*}
  a \equiv b \mod m & :\iff m | a-b \\
  a \equiv b (m)     & :\iff \bar{a} = \bar{b}
\end{align*}
\[
  \mathbb{Z}_m = \{ \bar{0},\bar{1},\bar{2}, \ldots, \overline{m-1}\} ,
  \bar{0} \cup \bar{1} \cup \bar{2} \cup  \ldots \cup \overline{m-1} = \mathbb{Z}
\]

\begin{definition}
\[
  \bar{a} + \bar{b} := \overline{a+b}
\]
\[
  \bar{a} \cdot \bar{b} := \overline{ab}
\]
\end{definition}

\Remark.
\begin{align*}
  a \equiv c(m), b \equiv d(m) \implies & a+b \equiv c+d(m) \\
                                        & ab \equiv cd (m)
\end{align*}
\begin{center}
\begin{minipage}[t]{0.2\textwidth}
    \begin{tabular}{c|cc}
      $+$             & $\overline{0}$  & $\overline{1}$\\
      \hline
      $\overline{0}$  & $\overline{0}$  & $\overline{1}$\\
      $\overline{1}$  & $\overline{1}$  & $\overline{0}$\\
    \end{tabular}
\end{minipage}
\begin{minipage}[t]{0.2\textwidth}
  \begin{tabular}{c|cc}
    $\cdot$         & $\overline{0}$  & $\overline{1}$\\
    \hline
    $\overline{0}$  & $\overline{0}$  & $\overline{0}$\\
    $\overline{1}$  & $\overline{0}$  & $\overline{1}$\\
  \end{tabular}
\end{minipage}
\end{center}

\Theorem.
$(\mathbb{Z}_m, + , \cdot)$ is a commutative ring with $1$

\begin{definition}
  We define the inverse element as follows:
\[
  \bar{a} \in \mathbb{Z}_m, \bar{x} \in \mathbb{Z}_m \text{ s.t }
  \bar{x} \cdot \bar{a} = \bar{1} \implies \bar{x} = \bar{a}^{-1}
\]
\end{definition}

\textbf{Example.}
\begin{align*}
  m = 5: &\quad \bar{2}^{-1} = \bar{3} \\
  m = 6: &\quad \bar{2}\cdot\bar{3} = \bar{0} \Rightarrow \bar{x} \cdot \bar{2} \cdot \bar{3} = \bar{0} \quad \bar{x} \cdot \bar{2} \neq \bar{1} \Rightarrow \nexists \bar{2}^{-1}
\end{align*}

\Theorem.
$\exists \bar{a}^{-1} \text{ in } \mathbb{Z}_m \iff gcd(a,m) = 1$ ($a,m$ are co-prime)

\Proof.

\ProofForward.
\begin{align*}
  & \bar{a}\cdot \bar{x} = \bar{1} \Rightarrow \exists k\in \mathbb{Z}:
    ax = 1 + km \Rightarrow ax - km = 1\\
  & d = gcd(a,m) \Rightarrow d \mid ax - km \Rightarrow d = 1
\end{align*}

\ProofBackward.
\begin{align*}
  & gcd(a,m) = 1 \Rightarrow \exists e,f : ae + mf = 1\\
  & \Rightarrow ae = 1 + (-f)m \Rightarrow \bar{a} \cdot \bar{e} = \bar{1}
    \Rightarrow \bar{e} = \bar{a}^{-1}
\end{align*}


\begin{definition}
Set of prime residue classes mod m
\[
  \mathbb{Z}_m^{*} = \{ \bar{a} \in \mathbb{Z}_m \mid gcd(a,m) = 1 \}
\]
This set contains all invertible elements of $\mathbb{Z}_m$. Hence, we can also define it as follows:
\[
  \mathbb{Z}_m^{*} = \{ x \in \mathbb{Z}_m \mid \exists x^{-1} : x \cdot x^{-1} = 1 \}
\]
This set is also referred to as \textbf{group of units}.
\end{definition}

\textbf{Example.}
\[
  \mathbb{Z}_5^{*} = \{ \bar{1}, \bar{2}, \bar{3}, \bar{4} \}, \mathbb{Z}_6^{*} = \{ \bar{1}, \bar{5} \}
\]

\textbf{Example.}
$ m = 17$, find $\overline{13}^{-1}$: $ 13 x \equiv 1 (17)$ $x \equiv 4 (17)$
\begin{align*}
  & 17 = 13 \cdot  1 +4\\
  & 13 = 4\cdot 3 +1 \iff 1 = 13-4\cdot 3 = 13-(17-13)\cdot 3 = 17\cdot
    3+4\cdot 13 \\
  & \Rightarrow x \equiv 4 (17)
\end{align*}

\textbf{Example.}
\begin{align*}
  3b &\equiv 3c (5) | \cdot 2 \\
  b &\equiv c(5)
\end{align*}

\textbf{Example.}
\begin{align*}
  3b &\equiv 3c (6) \\
  3b &= 3c + k \cdot 6 | :3 \\
  b &= c + k \cdot 2 \\
  b &\equiv c(5)
\end{align*}

\textbf{Example.}
\begin{align*}
  ab &\equiv ac (am) \Rightarrow b \equiv  c (m) \\
  ab &\equiv ac (m) \Rightarrow b \equiv c (m)
    \text{ if } ax \equiv 1 (m) \text{ has a solution } \iff gcd(a,m) = 1
\end{align*}


