
\section{Basics}

\subsection{Rings}
\begin{definition}
$(R,+,\cdot)$ is called a \dt{Ring} if:

\begin{center}
\begin{tabular}{ll}
  $(R,+)$ & abelian group (neutral element: 0)\\
  $(R,\cdot)$ & semigroup \\
  $\forall ab,c \in R:$ & $a(b+c) = ab+ac$\\
                        & $(a+b)c = ac+bc$\\
\end{tabular}
\end{center}
\end{definition}

\begin{definition}
An \dt{integral domain} is a \dt{commutative ring} with $1$ element and no zero divisors: $a\cdot b = 0 \implies a=0 \lor b = 0$
\end{definition}

\begin{definition}
An \dt{Euclidean ring} is an integral domain with an Euclidean function
\begin{align*}
  n: R \backslash \{0\} \rightarrow \mathbb{N} \text{ s.t. } &\forall a,b \in R: \exists q,r \in R: a = bq + r, \\
  &n(r) < n(b) \lor r=0 \\
  \text{ and } &n(a) \leq n(ab)
\end{align*}
\end{definition}

Let us assume we have an integral domain $(R,+, \cdot)$:\\
\[ t∣a \Leftrightarrow \exists c: a=tc =ct \]
\begin{center}
\begin{tabular}{ll}
$d=\gcd(a,b)$ if & 1) $d∣a \land d∣b$ \\
  & 2) $t∣a \land t∣b \implies t∣d$
\end{tabular}
\end{center}

\textbf{Remember}
in $\mathbb{Z}$: $d = gcd(a,b) \implies -d = \gcd(a,b)$

\begin{definition}
$R$ integral domain,
$a,b \in R$ are called associated, $a \sim b$,
\[
  \Leftrightarrow \exists \text{ unit } r \in R^{*}: a = rb
\]
\end{definition}

Recall the set (group) of units:
\[
  R^{*} = \{ x \in R \mid \exists x^{-1} : xx^{-1} = 1\}
\]
\begin{tabular}{llll}
  $(R^{*}, \cdot)$ \ldots group:
    & $1^{-1} = 1 $ & $\implies$ & $1 \in R^{*}$ \\
    & $x\in R^{*} $ & $\implies$ & $x^{-1} \in R^{*}$
\end{tabular}

\Theorem.
$R$ is an Euclidean ring, \[ a,b\in R, a∣b \implies n(a) \leq n(b) \]
\Proof.
\[
  \exists c : b = ac, n(a) \leq n(ac) = n(b)
\]

\textbf{Corollary.}
$d, d'$ are $\gcd$'s of $a$ and $b$ $\implies n(d) = n(d')$

\Proof.
$d∣d' \land d'∣d$


\Remark.
1) $x = a\cdot b, \quad a,b\not\in R^{*} \cup \{0\} \implies n(a) < n(x), n(b) < n(x)$

\Remark.
2) In general integral domains:
\[
  d= \gcd(a,b), d' = \gcd(a,b) \implies d \sim d'
\]
\begin{align*}
  &d∣d' \land d'∣d \implies d = c_1\cdot d', d' = c_2 \cdot d \\
  &\implies d = c_1c_2 d \implies d (1-c_1c_2) = 0 \implies c_1 c_2 = 1\\
  &\implies c_1,c_2 \in R^{*}
\end{align*}

\subsubsection{Generalization of Prime Numbers}
\begin{definition}
  Integral domain $R$, with $a\in R \backslash (\{0\}\cup R^{*})$, i.e. $\nexists a^{-1}$

$a$ \dt{irreducible} $\Leftrightarrow a = bc$ implies that either $b\in R^{*}$ or $c\in R^{*}$

\begin{tabular}{ll}
   Side remark: & In $\mathbb{Z}: \mathbb{Z}^{*} = \{-1, 1\}$\\
   & $p = p \cdot \mathbf{1} = (-p) \cdot (\mathbf{-1})$
\end{tabular}

$a$ \dt{prime element} $\Leftrightarrow a ∣ bc$ implies that $a∣b$ or $a∣c$
\end{definition}

\textbf{Example.}
$R = \mathbb{Z} \implies x \in R$ irreducible $\iff x \in \mathbb{P}$ or $-x \in \mathbb{P} \iff $ $x$ prime element

\Theorem.
\begin{enumerate}
  \item Every prime element is irreducible
  \item In Euclidean rings the converse is true as well
\end{enumerate}

\Proof.
\begin{enumerate}

  \item a is a prime element and $a = bc$ (which implies $a∣b$ or $a∣c$)\\\\
  If $a∣b$ then $a∣b \land b∣a \implies a = bc, b = a\bar{c} = b c\bar{c}$\\
  $\implies b(1-c\bar{c}) = 0 \implies c\bar{c} = 1 \implies c, \bar{c} \in R^{*}$\\
  $\implies a \sim b \implies$ a irreducible\\
  (because: a irreducible $\implies \underbrace{b \in R^{*}}_{a \sim c} \lor \underbrace{c \in R^{*}}_{a \sim b}$)

  \item a irreducible, $a∣bc$\\\\
  Suppose $a \nmid b: d = \gcd(a,b): a = d\cdot c_1,\; b = d\cdot c_2$.\\
  Either $d$ or $c_1$ has to be a unit.
  \begin{itemize}
    \item If $c_1 \in R^{*} \stackrel{\exists c_1^{-1}}{\implies} d = a\cdot c_1^{-1} \implies b = a \cdot c_1^{-1} \cdot c_2$\\
    $\implies a ∣b$ Contradiction!

    \item If $c_1 \not\in R^{*} \implies d \in R^{*}$:
    Without loss of generality: $d = 1$\\
    we are in an Euclidean Ring $R \implies \exists x,y \in R: 1 = ax + by$\\
    $\implies c = acx + bcy = acx + ay = a (cx+y) \implies a∣c$

  \end{itemize}
  In the same way we can proof $a \nmid c \implies a∣b$.

\end{enumerate}

\textbf{Example.}
$R = \mathbb{Z} [i \sqrt{5}] = \{a+bi \sqrt{5} \mid a,b \in \mathbb{Z} \} \subseteq \mathbb{C}, i = \sqrt{-1}$
\[(R,+,\cdot) \underbrace{\leq}_{\text{subring}} (\mathbb{C}, +,\cdot)\]
\begin{align*}
  (a+bi \sqrt{5}) (c+di\sqrt{5}) =
    \underbrace{ac - 5bd}_{\in \mathbb{Z}} + \underbrace{(ad+bc)}_{\in \mathbb{Z}} i\sqrt{5} \quad \in \mathbb{Z}[i\sqrt{5}]
\end{align*}
\begin{align*}
  1+0 \cdot i\sqrt{5} \in R\\
  6 = 2 \cdot 3 = (1+i\sqrt{5})(1-i\sqrt{5})
\end{align*}
\begin{align*}
  2∣6, 2 \nmid (1+i\sqrt{5}): \quad & 1+i\sqrt{5} = 2c = 2(a+bi\sqrt{5}) \\
    & \implies 2a = 1, 2b = 1 \\
    & \implies a \not\in \mathbb{Z}
\end{align*}

$2 \nmid (1 - i\sqrt{5})$ but $2 ∣ (1+i\sqrt{5})(1-i\sqrt{5})$
$\implies 2$ is not a prime element of $R$
\begin{align*}
  & 2 = \underbrace{(a+bi \sqrt{5})}_{r} \underbrace{(c+di\sqrt{5})}_{s}\\
  & \implies \frac{2}{a+ib\sqrt{5}} \frac{a-i\sqrt{5}b}{a-i\sqrt{5}b}
  = c+di\sqrt{5}
= \underbrace{\frac{2a}{a^2 + 5b^2}}_{\in \mathbb{Z}} - \underbrace{i\sqrt{5} \frac{2b}{a^2 + 5b^2}}_{\in \mathbb{Z}}
\end{align*}

But $a^2 + 5b^2 > 4b^2 > 2|b|$ except if $ b=0$ $\implies d \not\in \mathbb{Z}$ except if $b=0$ \\

If $b=0:$
\[ c = \frac{2a}{a^2} = \frac{2}{a} \implies a \in \{\pm 1, \pm 2\} \]
$\implies r = \pm 1 \implies r \in R^{*}$ or $ r = \pm 2 \implies s \in R^{*} \implies 2 $ irreducible

$\implies \mathbb{Z}[i\sqrt{5}]$ is not an Euclidean ring

\textbf{Example.}
$K$ is a field $\implies K[x]$ is a Euclidean ring\\

Definition of a field:
\begin{tabular}{ll}
  $(K,+)$ &abelian group \\
  $(K\backslash \{0\}, \cdot)$ & abelian group \\
  $a(b+c) = ab + ac$
\end{tabular}

These properties imply $\forall x \in K: 0\cdot x = 0$

Recall:
$n(p(x)) = \deg(p(x))$ and
$\deg(p(x)) \leq \deg(p(x) q(x))$.

$\implies$ The prime elements are the irreducible elements: irreducible polynomials: $a(x) = b(x)c(x) \implies \deg(b(x))=0 \lor \deg(c(x))=0$

Side remark:
\begin{align*}
  &r(x) \in K[x]^{*}, 1 = 1*x^0 + 0 x^1 + \ldots \\
  &r(x) \in K[x]^{*} \Leftrightarrow r(x) \neq 0, \deg(r(x)) = 0\\
\end{align*}
In $\mathbb{C}[x]:
\deg(p(x)) = n \implies \exists n$ not necessarily different zeros $a_1, a_2, \ldots , a_n$ and $p(x) = (x-a_1)(x-a_2) \ldots (x-a_n) \quad$ (fundamental theorem of algebra)

$\implies p(x)$ is irreducible $\iff$ $p(x) = ax+b$ linear polynomial

In $\mathbb{R}[x]:$
$x^4 + 1 = (x^2 + \sqrt{2} +1)(x^2 - \sqrt{2} + 1)$

\begin{tabular}{ll}
  irreducible:
    & $ax+b$ and \\
    & $ax^2 + bx + c$ without zeros\\
\end{tabular}

if you find a complex zero polynomial \\
$p(a) = 0 \implies p(\bar{a}) = 0$ \\
$(x-a)(x-\bar{a}) = x^2 - \underbrace{(a + \bar{a})}_{Re(a)} x + \underbrace{a\bar{a}}_{|a|^2}$

\begin{definition}
$R$ is an integral domain such that $\forall a \in R \backslash (\{0\} \cup R^{*})$
there exists a unique representation $a = \epsilon p_1 p_2 \ldots p_k$
where $\epsilon \in R^{*}$ and $p_1, \ldots, p_k$ are prime elements.

unique: ($\star$) $\epsilon p_1 \ldots p_k = \eta q_1 \ldots q_l \implies k=l$ and there is a permutation $\pi$ such that
$p_i \sim q_{\pi(i)} \quad \forall i = 1,\ldots, k \quad $($ p_i = \underbrace{\epsilon_i}_{\in R^{*}} q_{\pi(i)} $)

$15 = 1\cdot 3\cdot 5 = (-1) \cdot (-5) \cdot 3$, $ 3 \sim 3, 5 \sim (-5): 5=\underbrace{(-1)}_{\in\mathbb{Z}^{*}} (-5)$

Then $R$ is called a \dt{factorial ring} (dt.: ZPE-Ring).
\end{definition}

\Theorem.
Every Euclidean ring is a factorial ring.

\Proof.
We have to show existence and uniqueness.

\textbf{Existence:}
\begin{enumerate}[{Case} 1:]
  \item $a$ irreducible $\Leftrightarrow$ $a$ prime element $\implies$ $a=1\cdot a$. This is a representation as desired.

  \item $a= bc \quad b,c \not\in R^{*}$ $\implies$ $n(b) < n(a), n(c) < n(a)$

  Suppose that $a$ does not have a representation of the form ($\star$) and $n(a)$ is minimal.
  This implies that $b$ must have a prime representation $b = \epsilon_1 p_1 \ldots p_k$, $c = \epsilon_2 q_1 \ldots q_l$
  $\implies a = \underbrace{\epsilon_1 \epsilon_2}_{\in R^{*}} p_1 \ldots p_k q_1 \ldots q_l$. Contradiction!
\end{enumerate}

\textbf{Uniqueness:}
\begin{align*}
  & a = \epsilon p_1 \ldots p_k = \eta q_1 \ldots q_l \quad k\geq 2\\
  & \implies p_1 ∣ \eta q_1 \ldots q_l \underset{p_1 \not\in R^{*}}{\implies}{} p_1 ∣ q_1 \ldots q_l\\
  & \implies \exists i: p_1 ∣ q_i
\end{align*}
Without loss of generality $i = 1$:
\begin{align*}
  & p_1∣q_1 \implies p_1 \sim q_1 \implies p_1 = \epsilon_1 q_1 \text{ with } \epsilon_1 \in R^{*}\\
  & \implies\epsilon p_2 \ldots p_k = \underbrace{\eta \epsilon_1}_{\in R^{*}} q_2 \ldots q_l
\end{align*}
Without loss of generality: $p_2∣q_2$

If $l > k$: $\epsilon = \tilde{\epsilon} q_{k+1} \ldots q_l$. Contradiction!

This implies that $k=l$, which implies uniqueness.


\textbf{Remember from last time.}

There are rings, where we have unique factorizations into prime elements.
Every Euclidean ring is a factorial ring. Divisibility theory, Euclidean Algorithm.

We need to go deeper.

\subsubsection{Ideals in rings}

First some facts from proof-theory

\textbf{Recall.}
$(G,*)$ and subgroup $U \leq G: a * U = \{ a * x \mid x \in U \}$. These sets form a partition of the group.
$ a \neq a' \begin{cases} a*U = a'*U $ or$\\ (a*U) \cap (a'*U) = \varnothing \end{cases}$

\begin{tabular}{ll}
  left cosets  & $a*U$, $a \in G$\\
  right cosets & $U*a$
\end{tabular}


If $U \leq G$ such that $\forall a \in G: a*U = U*a$ then $U$ is called \dt{normal subgroup} $(U \NormSubgroup G)$

$U \NormSubgroup G :$ $\underbrace{(a*U)}_{=(a' *U)} * \underbrace{(b*U)}_{=(b' *U)} = \underbrace{(a*b) * U}_{=(a'*b')*U}$

$(G / U, *)$ is called a \dt{quotient group}

$G/U = \{ a*U \mid a \in G\}$ \quad (read: ``G modulo U'')

\begin{definition}
  $R,S$ ring, $\varphi : R \rightarrow S$

  (ring) homomorphism if
  \begin{align*}
    \varphi(a+b) &= \varphi(a) + \varphi(b)\\
    \varphi(a \cdot b) &= \varphi(a) \cdot \varphi(b)\\
  \end{align*}

  kernel of $\varphi$: $\Kernel \varphi = \{ x \in R \mid \varphi(x) = 0\}$
\end{definition}

\Theorem.
\begin{align*}
(\Kernel \varphi, +) \NormSubgroup (R, +) \text{ and } a \cdot \Kernel \varphi &\subseteq \Kernel \varphi\\
(\Kernel \varphi) \cdot a &\subseteq \Kernel \varphi
\end{align*}
\begin{align*}
  &x \in \Kernel \varphi \Rightarrow \varphi(ax) = \varphi(a) \underbrace{\varphi(x)}_{=0} = 0 \\
  & \Rightarrow a x \in \Kernel \varphi \text{, } x a \in \Kernel \varphi
\end{align*}

\begin{definition}
  $R$ ring, $I \subseteq R$ \dt{ideal} if
  \begin{enumerate}
    \item $(I, +)$ is (normal) subgroup of (R,+), which is an abelian group
    \item $a \cdot I \subseteq I$, $I \cdot a \subseteq I$
  \end{enumerate}
\end{definition}

The ``normal'' is in parentheses because every subgroup of an abelian group is a normal subgroup.

\Remark.
$\varphi: R \rightarrow S$ homomorphism $\implies$ $\Kernel \varphi$ is an ideal of $R$

\Theorem.
$R$ ring, $ I \subseteq R$ ideal. Then define $+$ and $\cdot$ on $\underbrace{R / I}_{=(R,+)/(I,+)}$ as follows:
\begin{align*}
  (a+I) + (b+I) &\coloneqq (a+b)+I, \\
  (a+I) \cdot (b+I) &\coloneqq (ab) + I
\end{align*}

Then $(R / I, +, \cdot)$ is a ring, the \dt{quotient ring} $R$ modulo $I$.

\textbf{Example.}
$R = \mathbb{Z}, I = n*\mathbb{Z} = \{n \cdot z \mid z \in \mathbb{Z}\}$ is an ideal
\begin{align*}
  U \subseteq G \Leftrightarrow & ~1)~ U \neq \varnothing\\
                                & ~2)~ a,b \in U \Rightarrow a*b^{-1} \in U\\
  x,y \in n\mathbb{Z} \implies  & x = k\cdot n, y = l\cdot n\\
                      \implies  & x -y = (k-l) \cdot n \in n\mathbb{Z}\\
                      \implies  & n\mathbb{Z} \NormSubgroup \mathbb{Z}:
                                  a \in \mathbb{Z} : a+n\mathbb{Z} =
                                  \{ a+nz \mid z\in \mathbb{Z} \}
\end{align*}
\begin{align*}
  a+n\mathbb{Z} = \bar{a} \text{ in } \mathbb{Z}_n \\
  a \cdot n \mathbb{Z} \subseteq n \mathbb{Z}
\end{align*}
\begin{align*}
  \underbrace{\mathbb{Z} / n \mathbb{Z}}_{=\mathbb{Z}_n}:
    & \bar{a} + \bar{b} = \overline{a+b}, \bar{a} \bar{b} = \overline{ab} \\
    & a+b \equiv a+b (n), a\cdot b \equiv a\cdot b (n)
\end{align*}

\Remark.
$R$ ring $\implies \{0\}$ and $R$ are the trivial ideals

\begin{tabular}{l|l}
+ & 0 \\
\hline
0 & 0
\end{tabular}
, $a \cdot \{0\} \subseteq \{0\}$

\begin{definition}
  $R$ ring, $I \subseteq R$ ideal

  We then define an equivalence relation $\sim$ on $R$ (i.e. $\sim \subseteq R \times R$)

  $\sim$ is compatible with $+$ and $\cdot$ if
  \begin{align*}
    a \sim b, c \sim d \implies & a +c \sim b+d \\
                                & a \cdot c \sim b \cdot d
  \end{align*}

  $\sim$ equivalence relation which is compatible with $+$ and $\cdot$
  is called \dt{congruence relation}, e.g $\equiv \mod n$

  In particular: $a\sim b :\Leftrightarrow a + I = b + I \Rightarrow \sim$ is a congruence relation
\end{definition}

\Theorem.
$R$ ring with $1$, $I \subseteq R$ ideal, $ \epsilon \in R^{*}, \epsilon \in I$

Then $R = I$

\Proof.
$\epsilon \in I \cap R^{*} \implies \exists \epsilon^{-1}$ \\
$\forall r \in R: r \cdot I \subseteq I$, in particular $\epsilon^{-1} \cdot \underbrace{I}_{\epsilon \in I} \subseteq I \implies 1 \in I$

$r \cdot I \subseteq I \implies r \cdot 1 = r \in I \implies R \subseteq I \implies R = I$

By definition we have $I \subseteq R$. In the proof we showed $R \subseteq I$. This implies $R = I$.

\Corollary.
A field $K$ has only the trivial ideals $\{0\}$ and $K$

\Remark.
$R$ ring, $(I_j)_{j\in J}$ family of ideals $\implies \bigcap_{j \in J} I_j$ is an ideal as well.

\begin{definition}
  $M \subseteq R$, $R$ ring,
  \[
    (M)\coloneqq \bigcap_{I \subseteq R, I \text{ ideal}, M \subseteq I} I
  \]
  is the \dt{ideal generated by M}. Which is the smallest ideal with contains $M$ (with respect to $\subseteq$).
\end{definition}

\begin{definition}
  An ideal generated by one element $a$ $\left( =(a) \right)$, is called a \dt{principal ideal}.
\end{definition}

\Theorem.
$R$ Euclidean ring. Then every ideal is a principal ideal.
i.e. $R$ is a \dt{principal ideal domain}.

\Remark.
Euclidean ring $\subseteq$ principal ideal domain $\subseteq$ factorial ring $\subseteq$ integral domain

\Example.
1) $R=\mathbb{Z}: (n) = n\cdot \mathbb{Z}$

$M = \{m_1, m_2, \ldots, m_k\} \subseteq \mathbb{Z}, (M)$\\
\begin{align*}
  m_1, m_2 &\implies a \cdot m_1 + b\cdot m_2 \in (M) \\
           &\implies \gcd(m_1,m_2) \cdot \mathbb{Z} \subseteq (M) \\
           &\implies M \subseteq gcd(m_1,m_2) \cdot \mathbb{Z} \\
           &\implies (M) = \gcd(m_1,m_2) \cdot \mathbb{Z}
\end{align*}

\Example.
2) $x\in R^{*} \implies (x) = R$

\Example.
3) $\mathbb{Q}$ is a subring (even a subfield) of $\mathbb{R}$, but no ideal

\subsection{Fields}

\textbf{Recall.} Properties of a Field:
\begin{itemize}
  \item $\underbrace{(K,+)}_{\text{neutral el. }= 0}$ is an abelian group

  \item  $\underbrace{(K\backslash \{0\}, \cdot)}_{\text{neutral el. }= 1}$ is an abelian group

  \item $\forall a,b,c \in K: a(b+c) = ab+ac, (a+b)c = ac+bc$

  \item $0\cdot x = (0+0) \cdot x = 0x + 0x \quad | -0x$\\
        $0 = 0x$\\
        $x0 = 0$ similarly
\end{itemize}

$(K_i)_{i\in I}$ family of subfields $\implies$ $\bigcap_{i\in I} K_i$ is a subfield

\begin{definition}
  $\displaystyle{\bigcap_{K' \text{ subfield of $K$}} K'}$ is \dt{prime field} of $K$, denoted by $P(K)$
\end{definition}

$\{0\}$ is not a field, because in every field we have that $0 \neq 1$ ($\implies$ every field has at least two elements)

\begin{definition}
  $\ord_{(K,+)} (1)$ is the \dt{characteristic of K} ($\Char K$)
\end{definition}

$1,1+1, 1+1+1, \ldots, \underbrace{1+1+ \ldots +1}_{\Char K} = 0$ if this is finite.

$\Char K = 0$ if $\ord_{(K,+)} (1) = \infty$

\Example.
\begin{align*}
  (\mathbb{R}, +): &\Char \mathbb{R} = 0\\
  &\Char \mathbb{Q} = 0\\
  &\Char \mathbb{C} = 0\\
  (\mathbb{Z}_2, +, \cdot): &\Char \mathbb{Z}_2 = 2\\
  &\Char \mathbb{Z}_p = p \text{ if } p \in \mathbb{P}
\end{align*}

\textbf{Properties of $P(K)$}
Case 1:
$\Char K=0$:
$\forall K'$ subfield of $K$: $0,1 \in K' \implies 0,1 \in P(K)$

\begin{align*}
&1, 1+1, 1+1+1, \ldots = k-1, k \in \mathbb{N}\\
&-1, (-1)+(-1), \ldots = k(-1) = -(k\cdot 1) = (-k) \cdot 1\\
&(k\cdot 1)^{-1} \in P(K)\\
&k\cdot 1, (-k)\cdot 1,(k\cdot 1)^{-1}, (k\cdot 1)\cdot (l\cdot 1)^{-1} \in P(K), \quad k \in \mathbb{Z}, l \in \mathbb{N} \backslash \{0\}\\
&\{ k \in \mathbb{Z}, l \in \mathbb{Z}, l>0 \} \cong \mathbb{Q}\\
&\implies P(K) \cong \mathbb{Q} \implies |K|=\infty
\end{align*}

Case 2: $\Char K \neq 0$

\Lemma.
\begin{align*}
  p = \ord_{(K,+)} (1) < ∞ \implies & \text{1) } ∀ a ∈ K ∖ \{0\}:
                                      \ord_{(K,+)} (a) = p\\
                                    & \text{2) } p \in \mathbb{P}
\end{align*}

\Proof.
\begin{enumerate}[1)]

  \item $p\cdot 1 = \underbrace{1+1+ \ldots +1}_{p \text{ times}} = 0$
  \begin{align*}
    \implies & p \cdot a = \underbrace{a+a+ \ldots +a}_{p\text{ times}} = a \cdot \underbrace{(1+1+ \ldots +1)}_{=0} = a\cdot 0 = 0 \\
    \implies & \ord(a) \leq p
  \end{align*}

  Assume $\ord(a) = m$
  \begin{align*}
    \implies
    & \overbrace{(m\cdot a)}^{=0}\cdot a^{-1} =
      \underbrace{\overbrace{a\cdot a^{-1}}^{=1} + \overbrace{a\cdot a^{-1}}^{=1} + \ldots + \overbrace{a\cdot a^{-1}}^{=1}}_{m\text{ times}} = m\cdot 1 \\
    \implies & m \geq p \implies m = p
  \end{align*}

  \item
  \begin{align*}
  p=a\cdot b & \implies 0=p \cdot 1 = \underbrace{\underbrace{1 + 1 + \ldots + 1}_{a\text{ times}} + a\cdot 1 + a\cdot 1 + \ldots + a\cdot 1}_{b\text{ times}}\\
  &\implies b \cdot (a \cdot 1) = a\cdot 1 + a\cdot 1 + \ldots + a\cdot 1\\
  &\implies \ord(a\cdot 1)=b < p \quad \text{Contradiction!}\\
  \Char K=p &\implies P(K) \cong \mathbb{Z}_p
  \end{align*}

\end{enumerate}


\textbf{Remember from last time.}
Fields,

Field $K$, $\Char K$, Prime field $P(K)$,

$P(K) \cong \begin{cases}$
$\mathbb{Q} \\$
$\mathbb{Z}_p, \quad p \in \mathbb{P}$
$\end{cases}$

\Corollary.
$K$ finite field. The characteristic cannot be $0$. We know that $\exists p \in \mathbb{P}, n \in \mathbb{N}^{+}: |K| = p^n$.

\Proof.
$P(K)$ finite, this means $\exists p \in \mathbb{P}$ such that $|P(K)| = p, P(K) \cong \mathbb{Z}_p$

$P(K) \subseteq K$: regard $K$ as vector space over $P(K)$. Scalars are taken from $P(K)$. We have finite spaces
\begin{align*}
  \implies & \exists \text{ base } \{ a_1, a_2, \ldots, a_n \} \subseteq K, \text{i.e.} \dim \GroupGenBy{K , P(K)} = n\\
  \implies & K= \left\{\sum_{i=1}^n \lambda_i a_i \mid \lambda_i \in \underbrace{P(K)}_{p\text{ elements}}, i = 1,2,\ldots,n\right\} \\
  \implies & |K| = p^n
\end{align*}

\Remark.
In fact, given $p,n \implies $

1) $\exists K: |K| = p^n$\\
2) $K, K'$ fields$: |K| = |K'| = p^n \implies K \cong K'$

$K$ field $\implies K[x]$ Euclidean ring $\implies$ every ideal is a principal ideal. Furthermore $K[x]$ is a factorial ring (unique factorization into primes).
\begin{align*}
  & P(x)= x^n + a_{n-1} x^{n-1} + \ldots + a_1 x_1 + a_0 \qquad a_0, a_1, \ldots ,a_{n-1} \in K \\
  & I = (P(x)) \text{ ideal in $K[x]$ } \implies P(x) \in I, Q(x)P(x) \in Q(x) \cdot I \subseteq I \\
  & \{R(x) \mid \exists Q(x) \in K[x] : P(x)Q(x) = R(x)\} = I
\end{align*}

$\underbrace{K[x]/ P(x)}_{\text{quotient ring}}$ ( i.e. $K[x] / (P(x))$ ): $A(x) \equiv B(x) \mod P(x) :\Leftrightarrow P(x) ∣ A(x) - B(x)$ $\implies \equiv $ is a congruence relation.
\begin{align*}
  A(x) \equiv B(x), C(x) \equiv D(X) \mod P(x) \\
  A(x) + C(x) \equiv B(x) + D(x) \mod P(x) \\
  A(x)C(x) \equiv B(x) D(x) \mod P(x)
\end{align*}

This implies, that we have a quotient ring $K[x]/ P(x)$

$P(x) \equiv 0 \mod P(x)$

$\implies x^n \equiv -a_{n-1}x^{n-1} - a_{n-2}x^{n-2} - \ldots - a_1 x - a_0 \mod P(x)$\\
each polynomial $Q(x)$ fulfills:
$Q(x) \equiv \tilde{Q}(x) \mod P(x), \deg \tilde{Q}(x)<n$

\Example.
$\mathbb{R}[x] / x^2-1$:
\begin{leftbar}
  Remark: $x^2 \equiv 1 \mod {x² -1}$
\end{leftbar}\vspace{-1cm}
\begin{align*}
  \underbrace{x⁴}_{\equiv x²}
  \underbrace{-3x³}_{\equiv -3x}
  \underbrace{+2x²}_{\equiv 2}
  -5x + 1                       & \equiv x² - 3x +2 - 5x + 1\\
                                & \equiv \underbrace{x²}_{\equiv 1} - 8x + 3\\
                                & \equiv -8x + 4\\
\end{align*}
$\mathbb{R}[x] / x^2 -1 = \{ \overline{ax+b} \mid a,b \in \mathbb{R} \}$

$\overline{x-1}, \overline{x+1}$ are zero-divisors $\implies \mathbb{R}[x] / x^2 -1$ is not an integral domain

In general: $K[x] / P(x) = \left\{ \overline{\sum_{i=0}^{n-1} b_i x^i} \mid b_i \in K\right\}$

If $P(x) = Q(x)R(x), \deg Q(x) \geq 1, \deg R(x) \geq 1 \implies K[x] / P(x)$ is not an integral domain.

\Theorem.
$K$ field, $P(x) \in K[x] \implies K[x] / P(x)$ is a field if and only if P(x) is irreducible.

\ProofForward.
Assume $K[x] / P(x)$ is a field. Then the polynomial $P(x)$ must be irreducible, otherwise $K[x] / P(x)$ has zero-divisors.

\ProofBackward. Assume $P(x)$ is irreducible:
$K[x] / P(x)$ is a commutative ring with 1-element $\bar{1} = 1+ (P(x))$

We have to show that every non-zero element has an inverse.

$A(x) \not\equiv 0 \mod P(x) \implies $ without loss of generality $\deg A(x) < \deg P(x) \implies \gcd(A(x), P(x)) = 1$\\
$\implies \exists B(x), C(x)$ such that
\begin{align*}
  & 1 = A(x)B(x) + P(x)C(x)\\
  \implies & 1 \equiv A(x) B(x) \mod P(x) \\
  \implies & B(x) = A(x)^{-1} \text{ in } K[x] / P(x) \\
  \implies & (K[x] / P(x))^{*} = (K[x]/P(x)) \backslash \{\bar{0}\} \\
  \implies & K[x] / P(x) \text{ is a field}
\end{align*}

Remember: $(K[x] / P(x))^{*}$ is the set of units in the ring

\Remark.
\begin{enumerate}[1)]
  \item $P(x)$ is irreducible, $\deg P \geq 2$\\
  $\implies P(x)$ has no zeros, since otherwise, say $P(a) = 0$, then $x - a ∣ P(x)$, which is a contradiction.

  \item $K$ subfield of $K[x]/P(x)$\\
  $K\hat{=}$ constant polynomials
\end{enumerate}

\subsubsection{Algebraic Extensions of a Field $k$}
$\mathbb{R}, ~ x^2 +1 = 0, ~ x = \pm \underbrace{\sqrt{-1}}_{i ∉ ℝ}$ \\
$\implies \mathbb{C} = \{a+bi \mid a,b \in \mathbb{R} \}$

$P(x)$ irreducible over $K$. $P(x) = 0$ has no solutions in $K$

$P(a) = 0 \implies a\notin K$, $ a \in L \supsetneqq K$

$P(x)$ is monic, i.e. $P(x) = x^n + a_{n-1}x^{n-1} + \ldots + a_1 x + a_0$ (the first coefficient is 1)

\Theorem.
If $K,L$ fields, $K\subseteq L$, $a$ is zero of some polynomial in $K[x]$

$a \notin K \implies \exists!$ monic and irreducible polynomial in $K[x]$ having $a$ as zero.

\Proof.
Existence: $K[x]$ is a factorial ring and an integral domain.

Uniqueness: Let us assume we have two polynomials $P_1(x) P_2(x)$, which are monic, irreducible, $P_1(x) \neq P_2(x), P_1(a) = P_2(a) = 0$
\begin{align*}
  \implies d(x) &= gcd(P_1(x), P_2(x))\\
                &= A(x) P_1(x) + B(x) P_2(x) \\
  \implies d(a) &= 0
\end{align*}

But we defined $d(x) = 1$, Contradiction!

$\implies P(x)$ is unique, $P(x)$ has minimal degree among all $Q(x)$ with $Q(a) = 0$. \\
$\implies$ If $P(x) = x^n + \sum_{i=0}^{n-1} p_i x^i \implies \sum_{i=0}^{n-1} p_i a^i \neq 0, \underbrace{\sum_{i=0}^{n-1} c_i a^i \neq 0}_{\hat{=} \sum_{i=0}^{n-1} c_i x^i \text{ in } \underbrace{K[x]/P(x)}_{\text{field!}}}$

$\underbrace{a^n + \sum_{i=0}^{n-1} p_i a^i = 0}_{\overline{x^n + \sum_{i=0}^{n-1} p_i x^i} = \bar{0}}$

$\implies L = \{ \sum_{i=0}^{n-1} c_i a^i \mid c_i \in K\}$ is the smallest field with $a \in L$. Moreover $L \cong K[x]/P(x)$

\begin{definition}
  $P(x)$ monic, irreducible over $K$, $\deg P(x) = n$, $P(a) = 0$,
  $L = \{ \sum_{i=0}^{n-1} c_i a^i\mid c_i \in K \} \supseteq K$

  $a$ \dt{algebraic} over $K$, $P(x)$ \dt{minimal polynomial of $a$ over $K$}, $L$ \dt{algebraic extension of $K$}, $L = K(a)$, ``$K$ adjoined $a$''
\end{definition}

\Example.
$\mathbb{C} = \mathbb{R}(i) \cong \mathbb{R}[x] / x^2 +1 = \{\overline{a+bx} \mid a,b \in \mathbb{R} \}$

\begin{align*}
  (a + bx)(c+dx)
  & \equiv ac + (ad+bc)x + bd \underbrace{x^2}_{\equiv -1} \\
  & \equiv ac - bd + (ad+bc)x
\end{align*}
\begin{align*}
  (a + bi)(c+di)
  & \equiv ac + (ad+bc)i + bd i^2 \\
  & \equiv ac - bd + (ad+bc)i
\end{align*}

\Example.
$\mathbb{Q}[x] / x^2 -2 \cong \mathbb{Q}(\sqrt{2}) = \{ a+b \sqrt{2} \mid a,b \in \mathbb{Q}\}$ \\
$x^2-2 = 0 \implies x = \pm \sqrt{2}$

$\sqrt{2}$ zero of the irreducible polynomial $x^2 -2 \implies x^2 = 2$

$\sqrt{2}$ irrational, algebraic

$\pi$ is not algebraic: $\nexists P(x) \in \mathbb{Q}[x]$ such that $P(\pi) = 0$. It is part of the transcendent numbers. Further examples are $\ln 2$ and $e$.

$\sqrt[n]{a}$ is algebraic: $x^n-a$

\Example.
$\underbrace{K[x]/ax+b}_{\cong K}$, $a,b \in K$, $ a\neq 0$ $\implies x = a^{-1}b$

\Example.
$\mathbb{Q}(\sqrt{2}): x^2 -3$ irreducible over $Q(\sqrt{2})$\\
$\implies \mathbb{Q}(\sqrt{2}, \sqrt{3})$

\Remark.
1) A maximal field has only irreducible polynomials of degree $1$. $\implies \nexists$ proper algebraic extensions $\rightarrow$ algebraically closed. Example: $\mathbb{C}$

2) If $K$ is a field then there exists a field $L$ such that $K \subseteq L$, $L$ algebraic closed.

3) If $\underbrace{|K| = p}_{\cong \mathbb{Z}_p}$ and $p\in \mathbb{P}$, then for every $n \in \mathbb{N}^{+}$ there is an irreducible polynomial $P(x) \in K[x]$. \\
$\implies \left| K[x]/P(x) \right| = \left| \{\overline{\sum_{i=1}^{n-1} c_i x^i} \mid c_i \in K\} \right| = p^n$

$\mathbb{Z}_p/P(x) \rightarrow$ the field of order $p^n$, Galois field $GF(p^n)$

\textbf{Proposition.}
Let $M(x)$ be the minimal polynomial of $a \in K$ and $f(x) \in K[x]$ such that $f(a) = 0$. Then $M(x)∣f(x)$. Obviously: If $g(x) = M(x)b(x) \implies g(a) = 0$

\Proof.
$f(x) = M(x) p(x) + q(x)$, $\deg q(x) < \deg M(x)$

$\underbrace{f(a)}_{=0} = 0 + q(a) \implies q(a) = 0 \implies q(x) = 0$

\textbf{Remember from last year.}
Finite Fields

\[
  \text{characteristic } K =
  \begin{cases}
    0                & \text{ if } \ord_{(K,+)} (1) = \infty \\
    \ord_{(K,+)} (1) & \text{ if } \ord_{(K,+)} (1) < \infty \\
  \end{cases}
\]

$K[x] / P(x)$ field $\leftrightarrow$ $P(x)$ irreducible

$P(x) \in K[x]$ monic, irreducible, $P(a) = 0 ( \Rightarrow a \not\in K)$

$\Rightarrow L = \{ \sum_{i=0}^{\deg(P)-1} c_i a^i \mid c_i \in K \}$,
$L = K(a) \cong K[x] / P(x)$

$P(x)$ minimal polynomial of $a$, $f(x) \in K[x], f(a) = 0 \Rightarrow P(x)∣f(x)$


